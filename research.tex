
{\LARGE \noindent  \textbf{Research Experience}}

\vspace{3mm}

\noindent I started my research experience three years ago with two undergraduate projects in physics and mathematics. These projects resulted in two monographs that gave me a solid background in topological quantum computing. After this, I felt prepared to start an original project for my masters, which is the experience I will mainly discuss in this text.   

\noindent \textbf{Masters Project :}  Manipulation of Majorana Modes in a Double Quantum Dot.

\noindent \textbf{Advisor:} Prof. Luis Gregorio Dias, University of Sao Paulo.

\noindent \textbf{Link:}	The main results of this project are so recent, that I cannot provide a completed manuscript. Instead, you can follow the progress of this paper in the following link: \url{https://git.io/fp2zX}

\noindent \textbf{Poster:} All the material presented here is summarized in a poster that I am attaching to this application. You may also find it in the link: \url{https://git.io/fp2vS} 

\noindent \textbf{Preliminary information:} A Majorana mode is a novel type of quasiparticle appearing at the edge of certain types of superconducting wires that undergo topological phase transitions \cite{kitaev_unpaired_2001}. The topological robustness of Majorana modes and their predicted non-abelian statistics make them a promising platform for quantum computers \cite{alicea_majorana_2010,alicea_new_2012,sarma_majorana_2015,lutchyn_majorana_2010}. So far, many experiments have confirmed the observation of Majorana signatures  but there is still skepticism about their existence \cite{mourik_signatures_2012,das_zero-bias_2012,deng_anomalous_2012,zhang_quantized_2018}.

\noindent \textbf{Overall description: }When a Quantum Dot (QD) is attached to the edge of a topological superconducting wire sustaining Majorana modes, the Majorana at the end of the wire tunnels inside the QD producing a decay of half a quanta in the QD conductivity \cite{liu_detecting_2011,vernek_subtle_2014,ruiz-tijerina_interaction_2015,lee_kondo_2013} . This is a Majorana signature that has been observed experimentally \cite{deng_majorana_2016}. This idea has turned on new lights into the design of quantum architectures \cite{karzig_scalable_2017}, because today’s precise experimental control over the parameters of QDs - energy levels, tunneling couplings, etc . - offers the unique possibility of manipulating the Majorana modes inside multi-dot systems. The simplest case where Majorana manipulation is possible is in a Double Quantum Dot. So far, no complete analysis of this simple case has been done. The goal of this  project is to fill this gap by realizing a full quantum transport study of the effects of coupling a Majorana mode with a Double Quantum Dot. 

\noindent \textbf{Methods:} We studied this system using the ballistic transport \cite{zubarev_double-time_1960} approach to obtain the exact Green functions for the case where Coulomb interaction is not relevant. We used NRG simulations  for the Coulomb interacting case \cite{wilson_renormalization_1975,krishna-murthy_renormalization-group_1980,bimberg_quantum_1999}. 

\noindent \textbf{Personal work:} I started this project with an NRG algorithm designed by my advisor. I modified it to include the new model which involved understanding a code based on object-oriented programming in C++. Due to the many subtleties of the model, which included particular Majorana symmetries and different connections with the dot, this part of the project took several months. 

By 2018, I was already producing results. Although the model looked simple, it had eight variable parameters taking into account all gate voltages and tunneling couplings, which gives more than 200 combinations. After collecting a large amount of data, one of the setups revealed the destruction of the Majorana signature under interference with one of the dots. This behavior was very intriguing. 

This discovery pushed me to obtain an exact theoretical description of the model, which required solving the Green functions of the system. One of the effects of having Majorana mode in the Hamiltonian is that it doubles the number of operators that are relevant in the transport equations. This elevates the complexity of the solution to such a point that solving the problem with traditional methods becomes unfeasible. Instead, I borrowed from tools of computer science for solving Laplacian systems \cite{spielman_algorithms_2010}. This allowed me to create a simple and graphical algorithm, that significantly simplified the solution of these equations. I really enjoyed this part of the project since I was able to showcase my mathematical creativity. 

\noindent \textbf{Results:} We were able to observe which dots presented Majorana signatures under different arrays of parameters. In the non-interacting case, we confirmed that shifting the QD’s gate voltage induces the Majorana to tunnel only to the other dot. In addition, an indirect coupling of the second dot could cause destructive interference with the Majorana signature. This was an unexpected and very exciting conclusion. The NRG simulations confirmed these results and showed that other interacting effects - like Kondo \cite{hewson_kondo_1997} and RKKY\cite{onsager_crystal_1944,yosida_magnetic_1957} - could coexist with the Majorana signatures. 

\noindent \textbf{Presentations:}  I presented partial advances of this project in the poster sessions of the congresses Autumn Meeting of the Brazilian Physical Society and Workshop on Strong Electron Correlations in Quantum Materials: Inhomogeneities, Frustration, and Topology. Recently, this project took a leading part of my advisor’s talk at Condensed Matter Theory in the Metropolis where I had an informal participation.

\noindent \textbf{Personal Accomplishments: } This project potentiated my numerical skills and strengthened my background in topological quantum computing and condensed matter physics. More importantly, it allowed me to understand what research is about. I learned that research leads to a wide variety of experiences. Sometimes the results are not the expected and the data does not makes sense. Sometimes that unexpected data turns into something new and exciting. Then, the best you can do is to enjoy the process,  keep a clear objective and continue working. I am confident that with this preparation I will be able to do many contributions to UBC research. Based on my background, I am prepared for theoretical and numerical work in different areas related to quantum computing, especially at its boundaries with condensed matter physics. 
