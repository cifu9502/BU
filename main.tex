%% start of file `template.tex'.
%% Copyright 2006-2013 Xavier Danaux (xdanaux@gmail.com).
%
% This work may be distributed and/or modified under the
% conditions of the LaTeX Project Public License version 1.3c,
% available at http://www.latex-project.org/lppl/.


\documentclass[11pt,a4paper,sans]{moderncv}        % possible options include font size ('10pt', '11pt' and '12pt'), paper size ('a4paper', 'letterpaper', 'a5paper', 'legalpaper', 'executivepaper' and 'landscape') and font family ('sans' and 'roman')
\usepackage{indentfirst}
\setlength{\parindent}{1.5cm}
% moderncv themes
\moderncvstyle{casual}    
% \usepackage{hyperref}
% style options are 'casual' (default), 'classic', 'oldstyle' and 'banking'
\moderncvcolor{blue}                               % color options 'blue' (default), 'orange', 'green', 'red', 'purple', 'grey' and 'black'
%\renewcommand{\familydefault}{\sfdefault}         % to set the default font; use '\sfdefault' for the default sans serif font, '\rmdefault' for the default roman one, or any tex font name
%\nopagenumbers{}                                  % uncomment to suppress automatic page numbering for CVs longer than one page
\usepackage{ragged2e}
% character encoding
\usepackage[utf8]{inputenc}           
% \usepackage{hyperref}
% \usepackage{hyperref}
% \hypersetup{
%     colorlinks,%
%     citecolor=blue,%
%     linkcolor=blue,%
%     urlcolor=blue
% }
% if you are not using xelatex ou lualatex, replace by the encoding you are using
%\usepackage{CJKutf8}                              % if you need to use CJK to typeset your resume in Chinese, Japanese or Korean

% adjust the page margins
\usepackage[scale=0.75]{geometry}
%\setlength{\hintscolumnwidth}{3cm}                % if you want to change the width of the column with the dates
%\setlength{\makecvtitlenamewidth}{10cm}           % for the 'classic' style, if you want to force the width allocated to your name and avoid line breaks. be careful though, the length is normally calculated to avoid any overlap with your personal info; use this at your own typographical risks...

% personal data
\name{Jes\'us David}{Cifuentes Pardo}
%\title{Resumé title}                               % optional, remove / comment the line if not wanted
%\address{street and number}{postcode city}{country}% optional, remove / comment the line if not wanted; the "postcode city" and and "country" arguments can be omitted or provided empty
%\phone[mobile]{+1~(234)~567~890}                   % optional, remove / comment the line if not wanted
%\phone[fixed]{+2~(345)~678~901}                    % optional, remove / comment the line if not wanted
%\phone[fax]{+3~(456)~789~012}                      % optional, remove / comment the line if not wanted
%\email{john@doe.org}                               % optional, remove / comment the line if not wanted
%\homepage{www.johndoe.com}                         % optional, remove / comment the line if not wanted
%\extrainfo{additional information}                 % optional, remove / comment the line if not wanted
%\photo[64pt][0.4pt]{picture}                       % optional, remove / comment the line if not wanted; '64pt' is the height the picture must be resized to, 0.4pt is the thickness of the frame around it (put it to 0pt for no frame) and 'picture' is the name of the picture file
%\quote{Some quote}                                 % optional, remove / comment the line if not wanted

% to show numerical labels in the bibliography (default is to show no labels); only useful if you make citations in your resume
\makeatletter
\renewcommand*{\bibliographyitemlabel}{\@biblabel{\arabic{enumiv}}}
\makeatother
\renewcommand*{\bibliographyitemlabel}{[\arabic{enumiv}]}% CONSIDER REPLACING THE ABOVE BY THIS
\renewcommand\refname{References}
% bibliography with mutiple entries
\usepackage{multibib}
\usepackage{cite}
\newcites{book,misc}{{Books},{Others}}
%----------------------------------------------------------------------------------
%            content
%----------------------------------------------------------------------------------


\begin{document}
%-----       letter       ---------------------------------------------------------
% recipient data
%\recipient{Company Recruitment team}{Company, Inc.\\123 somestreet\\some city}
\date{December 01, 2018}
\opening{Dear Admission Committee,}
\closing{Thank you for your consideration,}
%\enclosure[Attached]{curriculum vit\ae{}}          % use an optional argument to use a string other than "Enclosure", or redefine \enclname

%\makelettertitle

\justify
Dear Admission Committee:\\

\noindent “Factoring large numbers is tough!”. I have heard that many times. But I remember very well the first time I did. I was training for the math olympiads in high school. It was a number theory lecture, and we were learning about Fermat’s theorem.  “The best computer could attempt it for years without any chance of success”. It was really surprising for me. I learned to factor numbers in middle school, how could it be so hard? My math olympiads years left me many more memories and lessons, and I was very fortunate to participate in the International Mathematical Olympiad in Amsterdam. But that single phrase really got stuck in my head. Many years later, I found out that a quantum algorithm could factorize large numbers in a reasonable amount of time. And just like that, one of my childhood beliefs was shattered. I had to learn about this algorithm!. \\

By that time I was in the middle of my undergraduate studies in Physics and Mathematics at Universidad de Los Andes, the most prominent university in Colombia. The following years, quantum computing became my greatest interest. I started with Nielsen and Chuang’s book, reading about quantum algorithms and quantum information. Enrolling in seminars and congresses on the topic, I even got to understand Shor’s algorithm. Eventually, I had a breakthrough. It was in a fascinating lecture by Andrei Bernevig at my university where he presented a solution to one of the biggest challenges in quantum computing: high-decoherence. To address it, he proposed using Majorana quasiparticles appearing at the edges of topological superconductors. It was brilliant. Somehow topology seemed to trivially fix the decoherence-problem. Then they just needed to braid these Majoranas to encode the information and create quantum algorithms. Eureka. \\

This nice idea receives the name of topological quantum computing. I spent the rest of my undergraduate researching this field and wrote two theses projects. While my thesis in physics was centered on the main problems around Majorana fermions and topological materials, in my mathematics thesis I used an algebraic approach to understand topological quantum algorithms. These two projects boosted my abilities by solving numerical problems, using advanced mathematical tools, and then applying them to physics. Furthermore, the combination of approaches gave me a more comprehensive background on the topic. In particular, I learned that this “braiding” process was far more difficult than it seemed in Bernevig's lectures. But still, the idea was beautiful, the experiments were promising, the field was hot, and it was applicable to quantum computing. I was really looking forward to continue working on this topic. I found that opportunity at the University of São Paulo, consistently ranked as the best university in Latin America. \\

During the last two years, I have been in São Paulo pursuing a thesis-based masters degree.  My advisor is Prof. Luis Gregorio Dias from the Condensed Matter Department. Under his supervision, I have worked on a project inspired on the idea of coupling Majorana fermions with quantum dots. My intention is to explore the manipulability of Majorana fermions in a double-dot system. The last few months have been full of excitement as the simulations started to reveal some intriguing results. In turn, embarking me on the task of creating a theoretical model to explain these phenomena. It was difficult. The model was complex and quite entangled, but after months of hard work, I finally created an algorithm by borrowing some tools from computer science. This turned out to be a clever solution to the problem that can be applied to more general systems. Moreover, the model matched very well with the simulations, hence completing my project. One of the brightest points of this Master is that it allowed me to understand what research is about. I learned that research leads to a wide variety of experiences. Sometimes the results are not the expected and the data does not make sense. Sometimes that unexpected data turns into something new and exciting. Then, the best you can do is to enjoy the process,  keep a clear objective and continue working. Right now I am writing a paper presenting my results. I expect to send it for publication in the following months. \\

Aside from this research project, these years in Brazil have been an enriching experience. I have learned Portuguese and gotten a fantastic cultural exchange. In the academic field, I took advantage of the large condensed matter community in the country to participate in conferences, present my advances, and increase my knowledge of topological materials and solid state physics. These activities, paired with many hours of individual preparation, allowed me to improve my communication skills including my writings, computer designs, and presentations. I am using these skills right now while writing the paper, and I intend to use them at BU in my future papers, classes and conferences. Finally, this masters greatly enhanced my numerical skills as a result of my research project, and the additional training on machine learning, tensor networks, and NRG algorithms. \\



During this long path, I always kept track of the state-of-the-art of quantum computing getting to consider different approaches like quantum annealing and quantum simulation. The condensed matter faculty at BU outstands in these fields. Professors Claudio Chamon and Andrei Ruckenstein developed recently a major contribution to the area by creating a quantum simulation protocol for classical computation. In addition, I am very interested in the projects of professor Cristopher Lauman  and  Professor Gregg Jaeger which have interesting research lines in topological quantum computing and quantum information. I am also very excited by the proximity to the headquaters of outstanding quantum institutes like IBM and Google which provide an active labor market in quantum computing in the region. All of these opportunities make BU one of the best places to do quantum research.  I am confident that in this environment I will rapidly find a place where I could exploit my abilities and continue contributing to this area. After completing my Ph.D. at BU  I expect to  be fully prepared to research in this field at any top-level institution in the academia or the industry.\\

I have been in Boston a couple of times while visiting my brother. I got to enter BU in a wonderful bike trip were I was able to the admire the beautiful campus . I was gladly impressed by the wide diversity in Boston and I really felt more motivated to apply to BU when I learned about their remarkable efforts to foment a diversity and inclusion in the campus. I still remember my days in Boston as one of the best moments of my life, so I am now eager to come back this time as a graduate student to join the huge race to build a quantum computer. \\


Thank you for your consideration, 



\vspace{0.5cm}

\textbf{Jesús David Cifuentes Pardo}

%\makeletterclosing



%
{\LARGE \noindent  \textbf{Research Experience}}

\vspace{3mm}

\noindent I started my research experience three years ago with two undergraduate projects in physics and mathematics. These projects resulted in two monographs that gave me a solid background in topological quantum computing. After this, I felt prepared to start an original project for my masters, which is the experience I will mainly discuss in this text.   

\noindent \textbf{Masters Project :}  Manipulation of Majorana Modes in a Double Quantum Dot.

\noindent \textbf{Advisor:} Prof. Luis Gregorio Dias, University of Sao Paulo.

\noindent \textbf{Link:}	The main results of this project are so recent, that I cannot provide a completed manuscript. Instead, you can follow the progress of this paper in the following link: \url{https://git.io/fp2zX}

\noindent \textbf{Poster:} All the material presented here is summarized in a poster that I am attaching to this application. You may also find it in the link: \url{https://git.io/fp2vS} 

\noindent \textbf{Preliminary information:} A Majorana mode is a novel type of quasiparticle appearing at the edge of certain types of superconducting wires that undergo topological phase transitions \cite{kitaev_unpaired_2001}. The topological robustness of Majorana modes and their predicted non-abelian statistics make them a promising platform for quantum computers \cite{alicea_majorana_2010,alicea_new_2012,sarma_majorana_2015,lutchyn_majorana_2010}. So far, many experiments have confirmed the observation of Majorana signatures  but there is still skepticism about their existence \cite{mourik_signatures_2012,das_zero-bias_2012,deng_anomalous_2012,zhang_quantized_2018}.

\noindent \textbf{Overall description: }When a Quantum Dot (QD) is attached to the edge of a topological superconducting wire sustaining Majorana modes, the Majorana at the end of the wire tunnels inside the QD producing a decay of half a quanta in the QD conductivity \cite{liu_detecting_2011,vernek_subtle_2014,ruiz-tijerina_interaction_2015,lee_kondo_2013} . This is a Majorana signature that has been observed experimentally \cite{deng_majorana_2016}. This idea has turned on new lights into the design of quantum architectures \cite{karzig_scalable_2017}, because today’s precise experimental control over the parameters of QDs - energy levels, tunneling couplings, etc . - offers the unique possibility of manipulating the Majorana modes inside multi-dot systems. The simplest case where Majorana manipulation is possible is in a Double Quantum Dot. So far, no complete analysis of this simple case has been done. The goal of this  project is to fill this gap by realizing a full quantum transport study of the effects of coupling a Majorana mode with a Double Quantum Dot. 

\noindent \textbf{Methods:} We studied this system using the ballistic transport \cite{zubarev_double-time_1960} approach to obtain the exact Green functions for the case where Coulomb interaction is not relevant. We used NRG simulations  for the Coulomb interacting case \cite{wilson_renormalization_1975,krishna-murthy_renormalization-group_1980,bimberg_quantum_1999}. 

\noindent \textbf{Personal work:} I started this project with an NRG algorithm designed by my advisor. I modified it to include the new model which involved understanding a code based on object-oriented programming in C++. Due to the many subtleties of the model, which included particular Majorana symmetries and different connections with the dot, this part of the project took several months. 

By 2018, I was already producing results. Although the model looked simple, it had eight variable parameters taking into account all gate voltages and tunneling couplings, which gives more than 200 combinations. After collecting a large amount of data, one of the setups revealed the destruction of the Majorana signature under interference with one of the dots. This behavior was very intriguing. 

This discovery pushed me to obtain an exact theoretical description of the model, which required solving the Green functions of the system. One of the effects of having Majorana mode in the Hamiltonian is that it doubles the number of operators that are relevant in the transport equations. This elevates the complexity of the solution to such a point that solving the problem with traditional methods becomes unfeasible. Instead, I borrowed from tools of computer science for solving Laplacian systems \cite{spielman_algorithms_2010}. This allowed me to create a simple and graphical algorithm, that significantly simplified the solution of these equations. I really enjoyed this part of the project since I was able to showcase my mathematical creativity. 

\noindent \textbf{Results:} We were able to observe which dots presented Majorana signatures under different arrays of parameters. In the non-interacting case, we confirmed that shifting the QD’s gate voltage induces the Majorana to tunnel only to the other dot. In addition, an indirect coupling of the second dot could cause destructive interference with the Majorana signature. This was an unexpected and very exciting conclusion. The NRG simulations confirmed these results and showed that other interacting effects - like Kondo \cite{hewson_kondo_1997} and RKKY\cite{onsager_crystal_1944,yosida_magnetic_1957} - could coexist with the Majorana signatures. 

\noindent \textbf{Presentations:}  I presented partial advances of this project in the poster sessions of the congresses Autumn Meeting of the Brazilian Physical Society and Workshop on Strong Electron Correlations in Quantum Materials: Inhomogeneities, Frustration, and Topology. Recently, this project took a leading part of my advisor’s talk at Condensed Matter Theory in the Metropolis where I had an informal participation.

\noindent \textbf{Personal Accomplishments: } This project potentiated my numerical skills and strengthened my background in topological quantum computing and condensed matter physics. More importantly, it allowed me to understand what research is about. I learned that research leads to a wide variety of experiences. Sometimes the results are not the expected and the data does not makes sense. Sometimes that unexpected data turns into something new and exciting. Then, the best you can do is to enjoy the process,  keep a clear objective and continue working. I am confident that with this preparation I will be able to do many contributions to UBC research. Based on my background, I am prepared for theoretical and numerical work in different areas related to quantum computing, especially at its boundaries with condensed matter physics. 

%
{\LARGE \noindent  \textbf{Leadership Experience }}

\vspace{3mm} 

I was born into a loving family in Bogotá. My father died before I was born. But thanks to the efforts of my amazing mother, I always had everything I needed: food, health, education, and love. She worked very hard to be able to fund my studies. Her insatiable motivation encouraged me to strive for my own goals, and made me become one of the most promising students in Colombia. This allowed me to have outstanding participations in Scientific Olympiads. My greatest accomplishment was representing Colombia at the International Mathematical Olympiad (IMO) in Holanda. I also got other important prizes such as the first place at the National Physics Olympiad in 2011 and also at the Undergraduate Mathematical Olympiad in 2012. 

Years later I came back to the math olympiads, this time as an instructor. I helped train the new generations of Colombian mathematicians. This was my first contact with teaching. The students were 12 to 15 years old, a complicated age. After some practice, I learned how to engage their attention with activities such as games and conundrums.  I also learned how to control big groups of students and encourage them to study hard from an early age. I used my own life to motivate them. "If you strive hard for your goals, you will travel around the world", I told them. This experience shaped my incipient teaching and leadership skills, and allowed me to help my mother fund my undergraduate studies. 

The organization of congresses also enhanced my leadership skills. In 2013, I volunteered at the IMO organization in Colombia. Due to my previous experience as IMO participant, I was assigned as  leading guide in charge of a group of 5 colleagues. My tasks in the event included receiving and attending the Argentinian team, and coordinating my group in the organization of the exam. This last task demanded a vast amount of work during the two test days. I had to take extra-hours and motivate constantly my team. Finally, we were able to prepare the test-room for more than one thousand students that presented the IMO.  Years later, I also helped in the organization of the  \textit{Workshop on Mathematical Structure and Foundations of Quantum Physics} by assisting my undergraduate advisor -prof. Andrés Reyes-  with participants’ reception. In addition, I gave a lecture to encourage the other students to actively participate in the workshop. These experiences showed me the other side of events, all the hard work involved in their organization, but also, how rewarding it is when everything works perfectly. 
	 
These adventures motivated me to lead my own project in 2015, when I joined efforts with 3 colleagues to found the seminar of physics for students in my university. We aimed to create a space where senior undergraduates and graduate students could share their work. This offered a stress-free environment to talk about new topics in physics. Since there were no professors, the students felt more confident to raise their hands and ask questions. It was an utter success. In just a few months, our seminar became the most attended in the department. It introduced younger students to current research topics, prepared the lecturers to communicate their research, and became a space for students to learn about fellowships, summer programs and graduate programs. Furthermore, the seminar allowed us to establish connections with other universities in Colombia, especially with the National University. 

My specific role in the organization was to coordinate with other universities. It was challenging. I had to guarantee that, every week,  the burgeoning number of outside assistants were aware of the scheduling and could access the university. In addition, I encouraged the participants from the other universities to give talks, in order to maintain a weekly flow of lectures and to diversify research areas. This motivated the students of the National University to create a seminar of their own. In fact, they invited us to give some talks there. The student seminar is my greatest contribution to the shaping of new generations of scholars so far. Although I left Colombia two years ago, the seminar still exists. In fact, last December I gave a lecture there about my current research and experience in Brazil, opening up about my program to prospective students. They are now applying to the University of São Paulo.

Based on these previous experiences I am confident that I will be a great match for UBC.  I will use my communication skills to establish good relationships and collaborations between research groups. My leadership skills will help me organize prospective seminars, or participate in quantum startups. Finally, my strong commitment and drive will allow me to pursue my goal of participating in this race for the creation of a quantum computer.  


Thank you again for your consideration,
\vspace{0.7cm}

\textbf{Jesus David Cifuentes Pardo}

\bibliographystyle{unsrt}
%\bibliography{Majorana_DQD}


\end{document}


%% end of file `template.tex'.
