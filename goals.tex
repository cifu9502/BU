
{\LARGE \noindent  \textbf{Leadership Experience }}

\vspace{3mm} 

I was born into a loving family in Bogotá. My father died before I was born. But thanks to the efforts of my amazing mother, I always had everything I needed: food, health, education, and love. She worked very hard to be able to fund my studies. Her insatiable motivation encouraged me to strive for my own goals, and made me become one of the most promising students in Colombia. This allowed me to have outstanding participations in Scientific Olympiads. My greatest accomplishment was representing Colombia at the International Mathematical Olympiad (IMO) in Holanda. I also got other important prizes such as the first place at the National Physics Olympiad in 2011 and also at the Undergraduate Mathematical Olympiad in 2012. 

Years later I came back to the math olympiads, this time as an instructor. I helped train the new generations of Colombian mathematicians. This was my first contact with teaching. The students were 12 to 15 years old, a complicated age. After some practice, I learned how to engage their attention with activities such as games and conundrums.  I also learned how to control big groups of students and encourage them to study hard from an early age. I used my own life to motivate them. "If you strive hard for your goals, you will travel around the world", I told them. This experience shaped my incipient teaching and leadership skills, and allowed me to help my mother fund my undergraduate studies. 

The organization of congresses also enhanced my leadership skills. In 2013, I volunteered at the IMO organization in Colombia. Due to my previous experience as IMO participant, I was assigned as  leading guide in charge of a group of 5 colleagues. My tasks in the event included receiving and attending the Argentinian team, and coordinating my group in the organization of the exam. This last task demanded a vast amount of work during the two test days. I had to take extra-hours and motivate constantly my team. Finally, we were able to prepare the test-room for more than one thousand students that presented the IMO.  Years later, I also helped in the organization of the  \textit{Workshop on Mathematical Structure and Foundations of Quantum Physics} by assisting my undergraduate advisor -prof. Andrés Reyes-  with participants’ reception. In addition, I gave a lecture to encourage the other students to actively participate in the workshop. These experiences showed me the other side of events, all the hard work involved in their organization, but also, how rewarding it is when everything works perfectly. 
	 
These adventures motivated me to lead my own project in 2015, when I joined efforts with 3 colleagues to found the seminar of physics for students in my university. We aimed to create a space where senior undergraduates and graduate students could share their work. This offered a stress-free environment to talk about new topics in physics. Since there were no professors, the students felt more confident to raise their hands and ask questions. It was an utter success. In just a few months, our seminar became the most attended in the department. It introduced younger students to current research topics, prepared the lecturers to communicate their research, and became a space for students to learn about fellowships, summer programs and graduate programs. Furthermore, the seminar allowed us to establish connections with other universities in Colombia, especially with the National University. 

My specific role in the organization was to coordinate with other universities. It was challenging. I had to guarantee that, every week,  the burgeoning number of outside assistants were aware of the scheduling and could access the university. In addition, I encouraged the participants from the other universities to give talks, in order to maintain a weekly flow of lectures and to diversify research areas. This motivated the students of the National University to create a seminar of their own. In fact, they invited us to give some talks there. The student seminar is my greatest contribution to the shaping of new generations of scholars so far. Although I left Colombia two years ago, the seminar still exists. In fact, last December I gave a lecture there about my current research and experience in Brazil, opening up about my program to prospective students. They are now applying to the University of São Paulo.

Based on these previous experiences I am confident that I will be a great match for UBC.  I will use my communication skills to establish good relationships and collaborations between research groups. My leadership skills will help me organize prospective seminars, or participate in quantum startups. Finally, my strong commitment and drive will allow me to pursue my goal of participating in this race for the creation of a quantum computer.  


Thank you again for your consideration,
\vspace{0.7cm}

\textbf{Jesus David Cifuentes Pardo}
